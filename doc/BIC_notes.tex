\documentclass[letterpaper]{article}

\usepackage{amsmath}
\usepackage{hyperref}

\begin{document}

\title{Notes on Bayesian Information Criterion Calculation for X-Means Clustering}

\maketitle

This documents attempts to derive and explain the Bayesian Information
Criterion (BIC) as written in \cite{Pelleg2000}.

First an explanation of the notation: Indices $i$, $j$, \ldots will be
used for points.  $n$, $m$, \ldots will be used to index over cluster.
Some terms like $\mu$ and $D$ when subscripted by $n$ refer to the
value for cluster $n$ and when subscripted by a variable in
parentheses, e.g. $(i)$, refer to the value for the cluster to which
data point (e.g. $x_i$) belongs.

% I can replace "Pelleg and Moore" with a macro...
The Pelleg and Moore's x-means algorithm is built around the
``identical spherical assumption''.  The data is modeled by $n$
gaussians, each with identical variance $\sigma$ and differing
positions $\mu_n$.

Given that model, the probability of a data point $i$ located at $x_i$ is
%
\begin{equation}
  \label{eq:1}
  P(x_i) = \sum_{n=1}^K
               \underbrace{P(x_i \in D_n)}_{\substack{\
                  \text{prob.\ data point i is} \\
                  \text{an element of cluster $D_n$}}}
               \cdot \underbrace{P(x_i | x_i \in D_n)}_{\substack{\
                  \text{prob.\ element i is positioned} \\
                  \text{at $x_i$ if in cluster $D_n$}}}
\end{equation}
When computed under the maximum likelihood, the first term, $P(x_i
\in D_n)$, is
\begin{equation}
  \label{eq:2}
  P(x_i \in D_n) = \frac{R_n}{R} = \frac{R_{(i)}}{R}
\end{equation}
where $R_n$ is the number of elements in the cluster.\footnote{A
  fuller derivation can be found in \cite{DaumeIII2009}}

% The next step is to figure out the point probabilities under the
% maximum likelihood estimate $\hat{P}(x)$.  Assuming the clusters are
% spherical gaussians, the probability for the position $x_i$ in a
% cluster is

% \begin{equation}
%   \label{eq:9}
%   P(x_i) \propto \exp\left( - \frac{1}{2\sigma^2} \| x_i - \mu_{(i)} \|^2 \right)
% \end{equation}

% The constant of proportionality can be determined by computing the
% integral

% \begin{align}
%   \label{eq:10}
%   \int P(x) &\propto \int \exp\left( - \frac{1}{2\sigma^2} \| x - \mu \|^2 \right) dx \\
%             &= \int \exp\left( - \frac{1}{2\sigma^2} \sum_{\alpha=1}^M \left( x_{\alpha} - \mu_{\alpha} \right)^2 \right) \prod_{\alpha=1}^M dx_\alpha \\
%             &= \int \prod_{\alpha=1}^M \exp\left( -\frac{1}{2\sigma^2} \left( x_{\alpha} - \mu_{\alpha} \right)^2 \right) dx_\alpha \\
%             &= \prod_{\alpha=1}^M \int \exp\left( -\frac{1}{2\sigma^2} \left( x_{\alpha} - \mu_{\alpha} \right)^2 \right) dx_\alpha \\
%             &= \prod_{\alpha=1}^M \sqrt{2 \pi \sigma^2} \\
%             &= \left(2 \pi \sigma^2\right)^{M/2}
% \end{align}



The second term is a multivariate gaussian distribution centered at
$\mu_n$ and with variance $\sigma^2$,
%
\begin{equation}
  \label{eq:11}
  P(x_i|x_i \in D_n) = \frac{1}{(2 \pi \sigma^2)^{M/2}} \exp\left( -\frac{1}{2 \sigma^2} \| x_i - \mu_n \|^2\right)
\end{equation}

Combining \eqref{eq:1}, \eqref{eq:2}, and \eqref{eq:11}
%
\begin{align}
  \label{eq:12}
  P(x_i) & = \frac{R_n}{R} \frac{1}{(2 \pi \sigma^2)^{M/2}} \exp\left( -\frac{1}{2 \sigma^2} \| x_i - \mu_{(i)} \|^2\right) 
\end{align}

Converting to log-likelihoods,\footnote{The logarithms in all equations in this and \cite{Pelleg2000} are base-$e$.  Different bases can be used, but they
would lead to additional constant factors.}
%
\begin{align}
  l(D) & = \log \prod_i P(x_i) \nonumber \\
  & = \sum_i \log P(x_i) \nonumber \\
  \label{eq:13}
  & = \sum_i \left( \log \frac{R_n}{R}
                        + \log \left( \frac{1}{(2 \pi \sigma^2)^{M/2}} \right)
                        - \frac{1}{2\sigma^2} \|x_i - \mu_{(i)}\|^2 \right) \\
  & = \sum_{n=1}^K \sum_{x_i \in D_n} \left( \log \frac{R_n}{R}
                        + \log \left( \frac{1}{(2 \pi \sigma^2)^{M/2}} \right)
                        - \frac{1}{2\sigma^2} \|x_i - \mu_{(i)}\|^2 \right) \\
  & = \sum_{n=1}^K \left[ R_n \left( \log \frac{R_n}{R}
                                     - \frac{M}{2} \log \left(2 \pi \sigma^2 \right) \right)
                          - \frac{1}{2\sigma^2} \sum_{x_i \in D_n} \|x_i - \mu_{(i)}\|^2 \right]
\end{align}


In \cite{Pelleg2000} section 3.2, the authors incorrectly state the
maximum likelihood estimate for the variance for identical spherical
Gaussians is 
\begin{equation}
  \label{eq:20}
  \hat{\sigma}^2 = \frac{1}{R-K}\sum_{i} \| x_i - \mu_{(i)} \|^2
\end{equation}
where the hat above the $\sigma$ is used to denote the estimator.  The
actual maximum likelihood estimate for the variance is determined by
setting $\frac{\partial l(D)}{\partial \sigma}$ to 0.
\begin{align}
  \frac{\partial l(D)}{\partial \sigma} 
    &= \sum_{n=1}^K \left[ - \frac{M R_n}{\sigma}
       + \frac{1}{\sigma^3} \sum_{x_i \in D_n} \|x_i - \mu_{(i)}\|^2 \right] \nonumber \\
    &= \sigma^{-3} \sum_{n=1}^K \left[ -M R_n\sigma^2
       + \sum_{x_i \in D_n} \|x_i - \mu_{(i)}\|^2 \right] \nonumber \\
  \label{eq:21}
    &= \sigma^{-3} \left[ -M R \sigma^2
       + \sum_i \|x_i - \mu_{(i)}\|^2 \right]
\end{align}
which is 0 when

\begin{equation}
  \label{eq:19}
  \hat{\sigma}^2 = \frac{1}{M R} \sum_{i} \|x_i - \mu_{(i)} \|^2
\end{equation}
Again using a hat to denote the variance estimate.\footnote{A more
  detailed derivation of the maximum likelihood estimate is given in
  \cite{DaumeIII2009}.}

Both estimates for the variance are proportional to the sum of the distances
from each data point to the nearest centroid.  This sum can be broken
down into the sum by cluster,

\begin{equation}
  \label{eq:3}
  \sum_i \|x_i - \mu_{(i)}\|^2 = \sum_n \sum_{i \in D_n} \|x_i - \mu{(i)} \|^2
\end{equation}

Each individual cluster is a spherical gaussian, which means
the unbiased estimator of the variance is
\begin{equation}
  \label{eq:4}
  \hat{\sigma}_n^2 = \frac{1}{M(R_n-1)}\sum_{i \in D_n} \| x_i - \mu_n \|^2
\end{equation}

Substitution of \eqref{eq:4} into \eqref{eq:3} yields
\begin{equation}
  \label{eq:5}
  \sum_i \|x_i - \mu_{(i)}\|^2 = M \sum_n \left(R_n-1 \right) \hat\sigma_j^2
\end{equation}

Assuming the ``identical spherical assumption'' means all the
Gaussians have the same variance
\begin{equation}
  \label{eq:6}
  \hat{\sigma}_j^2 = \hat{\sigma}^2
\end{equation}
then \eqref{eq:5} becomes
\begin{align}
  \nonumber
  \sum_i \|x_i - \mu_{(i)}\|^2 &= M \left(\sum_n \left(R_n\right) - \sum_n\left( 1 \right)\right) \hat\sigma^2 \\
  \label{eq:7}
             &= M \left( R - K \right) \hat{\sigma}^2
\end{align}
or, 
\begin{equation}
  \label{eq:8}
  \hat{\sigma}^2 = \frac{1}{M(R-K)}\sum_{i} \| x_i - \mu_{(i)} \|^2
\end{equation}
which is the same as \eqref{eq:20} up to a factor of $M$.  So Pelleg
and Moore are using the unbiased estimator for each cluster.  Now
using the maximum likelihood assumption from \eqref{eq:5},
%
\begin{align}
  \label{eq:14}
  \hat{l}(D) &= \sum_{n=1}^K
                   \left[ R_n \left( \log \frac{R_n}{R}
                          - \frac{M}{2} \log \left(2 \pi \hat{\sigma}^2 \right) \right)
                   - \frac{1}{2\hat{\sigma}^2} M \left( R_n - 1 \right) \hat{\sigma}^2 \right] \\
             &= \sum_{n=1}^K \left[ R_n \log R_n - R_n \log R - \frac{R_nM}{2} \log \left(2 \pi \hat{\sigma}^2 \right)
                   - \frac{1}{2} M \left( R_n - 1 \right)\right]
\end{align}

Using $\sum_{n=1}^K R_n = R$,

\begin{align}
  \label{eq:15}
  \hat{l}(D) &= \sum_{n=1}^K R_n \log R_n - R \log R - \frac{RM}{2} \log \left(2 \pi \hat{\sigma}^2 \right)
                   - \frac{M}{2} \left( R - K \right)
\end{align}


Now, consider two hypothesis, $\phi_1$ and $\phi_2$ (denoted $M_j$ in
the article, but I want to be clear this has nothing to do with the
number of dimensions $M$).  $K$, $R_n$, and $\sigma$ are all functions
of the models, $\phi$.  In our case, $\phi_1$ is the clustering result
after minimizing with a fixed number of clusters, and $\phi_2$ is the
result after splitting one of the clusters into two and doing k-means
only over that original cluster.  Adding more clusters will decrease
the variance (and increase the likelihood).  (Or it could leave it
unchanged, if the additional cluster is empty.)  To avoid constantly
adding centroids, information criteria are used.  The main ones are
the Akaike information criterion, and the one used in
\cite{Pelleg2000}, the Bayesian information
criterion.\footnote{Actually this is the Schwarz information criterion
  used as an approximation for a Bayes factor, only valid when the
  number of points under consideration is large.  See \cite{Kass1995}
  for details.}
\begin{equation}
  \label{eq:22}
  BIC(\phi) = \hat{l}_\phi(D) - \frac{p_{\phi}}{2} \cdot \log R
\end{equation}
Both work by penalizing the likelihood more as the complexity of the
model (e.g.\ the number of parameters) increases.  So, instead of
checking the likelihoods only, hypothesis $\phi_2$ is better than
hypothesis $\phi_1$ if $BIC(\phi_2) > BIC(\phi_1)$.

For clarity, the maximum likelihood can be broken into the sum of two
parts: a model-dependent part and a model-independent part.
\begin{align}
  \label{eq:16}
  \hat{l}(D, \phi) &= \sum_{n=1}^{K(\phi)} R_n(\phi) \log R_n(\phi) - R \log R - \frac{RM}{2} \log \left(2 \pi \hat{\sigma}(\phi)^2 \right)
                   - \frac{M}{2} \left( R - K(\phi) \right) \\
                   &= \left[ \sum_{n=1}^{K(\phi)} R_n(\phi) \log R_n(\phi) + \frac{M K(\phi)}{2} - \frac{RM}{2} \log \left( \hat{\sigma}(\phi)^2 \right) \right] \nonumber \\
                   & \quad - \left[ \frac{M R}{2} + R \log R + \frac{RM}{2} \log 2 \pi \right] \\
                   &= \hat{l}_{\text{model-dependent}}(D, \phi) + \hat{l}_{\text{model-independent}}(D)
\end{align}

Using the definition of the $BIC$ and eliminating the model-independent terms,
\begin{align}
  \hat{l}(D, \phi_2) - \frac{p_{\phi_2}}{2} \log R &> \hat{l}(D, \phi_1) - \frac{p_{\phi_1}}{2} \log R \nonumber \\
  \label{eq:17}
  \hat{l}_{\text{model-dependent}}(D, \phi_2) - \frac{p_{\phi_2}}{2} \log R &> \hat{l}_{\text{model-dependent}}(D, \phi_1) - \frac{p_{\phi_1}}{2} \log R \nonumber \\
\end{align}

This gives a final test of the form
\begin{multline}
  \label{eq:18}
  \left[ \sum_{n=1}^{K(\phi_2)} R_n(\phi_2) \log R_n(\phi_2) + \frac{M K(\phi_2)}{2} - \frac{RM}{2} \log \left( \hat{\sigma}(\phi_2)^2 \right) \right] - \frac{p_{\phi_2}}{2} \log R \\
   > \left[ \sum_{n=1}^{K(\phi_1)} R_n(\phi_1) \log R_n(\phi_1) + \frac{M K(\phi_1)}{2} - \frac{RM}{2} \log \left( \hat{\sigma}(\phi_1)^2 \right) \right] - \frac{p_{\phi_1}}{2} \log R
\end{multline}
%
If this inequality holds, $\phi_2$ is considered a better model the $\phi_1$.

\bibliography{BIC_notes}{}
\bibliographystyle{plainnat}

\end{document}
